\documentclass[12pt,a4paper]{article}
\usepackage[utf8]{inputenc}
\usepackage{amsmath}
\usepackage{amsfonts}
\usepackage{amssymb}
\usepackage{lipsum}
\usepackage{textcomp}

\usepackage{makecell} % linebreak dans une cellule
\usepackage{multicol} % twocols localement
\usepackage{vwcol} % idem mais avec largeur variable
\usepackage{color, colortbl} % colorer les tableaux
\usepackage{enumitem} % utiliser des lettres pour énumérer
\usepackage{wrapfig} % insérer des images dans dutexte
\usepackage{dashundergaps} % transformer du texte en ________
\usepackage{MnSymbol,wasysym} % smileys
\usepackage{minibox} % multiline fbox - \minibox[frame]{}
\usepackage[pscoord]{eso-pic} % floating text box \placetextbox{x}{y}{text}
\usepackage{ifthen}
\usepackage{soul} % teste barré \st

% --- geometry ---
\usepackage{geometry}
\geometry{legalpaper, margin=1cm}
% ---

% --- xcolor ---
\usepackage{xcolor}
\definecolor{lightgray}{gray}{0.9}
% ---

% --- tcolorboxes ---
\usepackage[most]{tcolorbox}
\newtcolorbox{definition}[2][]{%
  attach boxed title to top left
               = {yshift=-8pt},
  colback      = white,
  colframe     = gray,
  fonttitle    = \bfseries,
  colbacktitle = gray,
  title        = #2,#1,
  enhanced,
}
% ---

% --- array ---
\usepackage{array}
\newcolumntype{L}[1]{>{\raggedright\let\newline\\\arraybackslash\hspace{0pt}}m{#1}}
\newcolumntype{C}[1]{>{\centering\let\newline\\\arraybackslash\hspace{0pt}}m{#1}}
\newcolumntype{R}[1]{>{\raggedleft\let\newline\\\arraybackslash\hspace{0pt}}m{#1}}
 % ---


\renewcommand{\baselinestretch}{1.15} % augmenter l'interligne

\dashundergapssetup{
	teacher-gap-format=underline,
	gap-widen
}

\newcommand{\placetextbox}[3]{% \placetextbox{<horizontal pos>}{<vertical pos>}{<stuff>}
  \setbox0=\hbox{#3}% Put <stuff> in a box
  \AddToShipoutPictureFG*{% Add <stuff> to current page foreground
    \put(\LenToUnit{#1\paperwidth},\LenToUnit{#2\paperheight}){\vtop{{\null}\makebox[0pt][c]{#3}}}%
  }%
}%


\author{Paul Clavier}
\title{Chapitre 5 - Droites parallèles et perpendiculaires - Interrogation} 

\begin{document}

% --- Section & subsection renum ---
\renewcommand\thesection{\Roman{section}}
\renewcommand\thesubsection{\arabic{subsection}}
% ---

% --- Selection manuelle de la version ---
%\def\isprof{true}
% ---

% --- Selection automatique de la version ---
\ifdefined\isprof
	\TeacherModeOn
\fi

% ---

\begin{center}
	\minibox[frame,c]{Chapitre 5 - Droites parallèles et perpendiculaires \\ Interrogation Écrite}
\end{center}

\placetextbox{0.1}{0.99}{Nom:}
\placetextbox{0.1}{0.96}{Prénom:}

\begin{center}
\begin{tabular}{|l|c|}
\hline \rowcolor{lightgray}
Droites parallèles et perpendiculaires \hspace{8cm} & Maitrise \\ \hline
\thead[l]{D1.3:2.1: Utiliser des notions de géométrie plane} &
\\ \hline
\thead[l]{D2:1.5: Présenter son travail de façon soignée} &
 \\ \hline
\end{tabular}
\end{center}

\textbf{Exercice 1}\\

\begin{minipage}{0.45\textwidth}
\begin{center}
Construit la droite (d') perpendiculaire à (d) passant par M.\\
\ifdefined\isprof
 	\includegraphics[scale=1]{img/eval-perp-corr.png}
 \else
  	\includegraphics[scale=1]{img/eval-perp.png}
 \fi 
\end{center}
\end{minipage}
\hfill\vline\hfill
\begin{minipage}{0.45\textwidth}
\begin{center}
Construit la droite (d') parallèle à (d) passant par P.\\
\ifdefined\isprof
	\includegraphics[scale=1]{img/eval-para-corr.png} 
\else
	\includegraphics[scale=1]{img/eval-para.png} 
\fi
\end{center}
\end{minipage}
\textbf{Exercice 2}
\begin{enumerate}
\item Trace $(d)$ et $(d')$ séantes en $I$. $(d)$ et $(d')$ ne doivent pas être perpendiculaires.
\item Place le point $A\in (d)$ tel quel $AI = 3cm$
\item Place le point $B\in (d')$ tel que $BI = 4cm$
\item Trace $(f)\perp (d')$ passant par $A$
\item Trace $(h)\parallel (d)$ passant par $B$
\item Place $J$, le point d'intersection de $(f)$ et $(h)$
\item Trace $[IJ)$
\end{enumerate}
\ifdefined\isprof
	\includegraphics[scale=1]{img/eval-2-corr.png} 
\fi

\end{document}