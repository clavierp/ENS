\documentclass[12pt,a4paper]{article}
\usepackage[utf8]{inputenc}
\usepackage{amsmath}
\usepackage{amsfonts}
\usepackage{amssymb}
\usepackage{lipsum}
\usepackage{textcomp}

\usepackage{makecell} % linebreak dans une cellule
\usepackage{multicol} % twocols localement
\usepackage{vwcol} % idem mais avec largeur variable
\usepackage{color, colortbl} % colorer les tableaux
\usepackage{enumitem} % utiliser des lettres pour énumérer
\usepackage{wrapfig} % insérer des images dans dutexte
\usepackage{dashundergaps} % transformer du texte en ________
\usepackage{MnSymbol,wasysym} % smileys
\usepackage[thinlines]{easytable} % tableaux à taille fixe

% --- geometry ---
\usepackage{geometry}
\geometry{legalpaper, margin=2cm}
% ---

% --- xcolor ---
\usepackage{xcolor}
\definecolor{lightgray}{gray}{0.9}
% ---

% --- tcolorboxes ---
\usepackage[most]{tcolorbox}
\newtcolorbox{definition}[2][]{%
  attach boxed title to top left
               = {yshift=-8pt},
  colback      = white,
  colframe     = gray,
  fonttitle    = \bfseries,
  colbacktitle = gray,
  title        = #2,#1,
  enhanced,
}
% ---


\renewcommand{\baselinestretch}{1.15} % augmenter l'interligne

\dashundergapssetup{
	teacher-gap-format=underline,
	gap-widen
}



\author{Paul Clavier}
\title{Chapitre 0 - Activité d'introduction Numérations Anciennes}

\begin{document}

% --- Section & subsection renum ---
\renewcommand\thesection{\Roman{section}}
\renewcommand\thesubsection{\arabic{subsection}}
% ---

% --- Selection manuelle de la version ---
%\TeacherModeOn
% ---

% --- Selection automatique de la version ---
\ifdefined\isprof
	\TeacherModeOn
\fi

% ---


% --- Définitions locales ---
\newcommand{\clou}{\includegraphics[scale=0.5]{img/bab-cloud.png}}
\newcommand{\chevron}{\includegraphics[scale=0.5]{img/bab-chevron.png}}
% ---


\begin{center}
	\fbox{\huge Activité d'introduction Numérations Anciennes}
\end{center}

\section{La numération dans l'antiquité}
\subsection{Les Babyloniens}

\begin{minipage}{0.2\textwidth}
\includegraphics[scale=0.8]{img/bab-pierre.png} 
\end{minipage}
\begin{minipage}{0.8\textwidth}

Chez les Babyloniens (environ 2000 ans av.J.C.), les symboles utilisés sont le clou \includegraphics[scale=0.5]{img/bab-cloud.png} pour l’unité le chevron \includegraphics[scale=0.5]{img/bab-chevron.png} pour les dizaines. Le système est additif jusqu’à et 60. 
Cette numération est dite de position, car le nombre dépend de la position des symboles utilisés.
\begin{center}
\begin{tabular}{|c|c|c|c|}
\hline
2 & 9 & 12 & 53 \\ \hline
\clou\clou & \clou\clou\clou\clou\clou\clou\clou\clou\clou & \thead{\chevron\clou\clou} & \thead{\chevron\chevron\clou\clou\clou}
\\ \hline
\end{tabular}
\end{center}
A partir de 60, la position des symboles entre en jeu de la façon suivante : 
\end{minipage}

\begin{center}
\begin{tabular}{|c|c|@{}m{0cm}@{}}
\hline
171 & \hspace{4cm}248\hspace{4cm} & \\ \hline
$2\times 60 + 51$ & \gap*[b]{$4\times 60 + 8$} & \thead{\vspace{0.5cm}} \\ \hline
\clou\clou\hspace{0.5cm}\chevron\chevron\chevron\chevron\chevron\clou & \gap*[b]{\clou\clou\clou\clou\hspace{0.5cm}\clou\clou\clou\clou\clou\clou\clou\clou} & \thead{\vspace{1cm}}\\\hline
\end{tabular}
\end{center}
\textbf{Exercice}: Écrire 18, 44 et 555 en numération babylonienne
\begin{center}
\begin{tabular}{|c|c|c|@{}m{0cm}@{}}
\hline
18 & 44 & 555\\ \hline
	\gap*[b]{\chevron\clou\clou\clou\clou\clou\clou\clou\clou} & 
	\gap*[b]{\chevron\chevron\chevron\chevron\clou\clou\clou\clou} &
	\gap*[b]{\clou\clou\clou\clou\clou\clou\clou\clou\clou \hspace{0.5cm} \chevron\clou\clou\clou\clou\clou } & \thead{\vspace{1cm}} \\ \hline
\end{tabular}
\end{center}

\subsection{Les Égyptiens}
Les scribes égyptiens (environ 1800 av. J.C.) représentaient 1 et les multiples de 10 par un croquis
\begin{center}
\begin{tabular}{|c|c|c|c|c|c|c|}
\hline
1 & 10 & 100 & 1 000 & 10 000 & 100 000 & 1 000 000 \\\hline
\includegraphics[scale=0.5]{img/egypt-1.png} & 
\includegraphics[scale=0.5]{img/egypt-10.png}  & 
\includegraphics[scale=0.5]{img/egypt-100.png}  & 
\includegraphics[scale=0.5]{img/egypt-1k.png}  & 
\includegraphics[scale=0.5]{img/egypt-10k.png}  & 
\includegraphics[scale=0.5]{img/egypt-100k.png}  & 
\includegraphics[scale=0.5]{img/egypt-1M.png} \\ \hline
& & Corde enroulée & Fleur de lotus & Doigt coupé & Têtard & Dieu assis\\ \hline
\end{tabular}
\end{center}
Pour lire un nombre, on additionne la valeur de l’ensemble des symboles utilisés dans son écriture:\\
\begin{tabular}{l l}
\thead{\includegraphics[scale=0.5]{img/egypt-345.png}} &  Signifie 5 + 40 + 300 = 345\\
\end{tabular}\\
\textbf{Exercice}: Écrire 1 342 422 en numération égyptienne.\\
\gap*[b]{\includegraphics[scale=0.3]{img/egypt-1.png}\includegraphics[scale=0.3]{img/egypt-1.png}\includegraphics[scale=0.3]{img/egypt-10.png}\includegraphics[scale=0.3]{img/egypt-10.png}\includegraphics[scale=0.3]{img/egypt-100.png}\includegraphics[scale=0.3]{img/egypt-100.png}\includegraphics[scale=0.3]{img/egypt-100.png}\includegraphics[scale=0.3]{img/egypt-100.png}\includegraphics[scale=0.3]{img/egypt-1k.png}\includegraphics[scale=0.3]{img/egypt-1k.png}\includegraphics[scale=0.3]{img/egypt-10k.png}\includegraphics[scale=0.3]{img/egypt-10k.png}\includegraphics[scale=0.3]{img/egypt-10k.png}\includegraphics[scale=0.3]{img/egypt-10k.png}\includegraphics[scale=0.3]{img/egypt-100k.png}\includegraphics[scale=0.3]{img/egypt-100k.png}\includegraphics[scale=0.3]{img/egypt-100k.png}\includegraphics[scale=0.3]{img/egypt-1M.png}}
\end{document}