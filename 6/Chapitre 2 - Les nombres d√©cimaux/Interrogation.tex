\documentclass[12pt,a4paper]{article}
\usepackage[utf8]{inputenc}
\usepackage{amsmath}
\usepackage{amsfonts}
\usepackage{amssymb}
\usepackage{lipsum}
\usepackage{textcomp}

\usepackage{makecell} % linebreak dans une cellule
\usepackage{multicol} % twocols localement
\usepackage{vwcol} % idem mais avec largeur variable
\usepackage{color, colortbl} % colorer les tableaux
\usepackage{enumitem} % utiliser des lettres pour énumérer
\usepackage{wrapfig} % insérer des images dans dutexte
\usepackage{dashundergaps} % transformer du texte en ________
\usepackage{MnSymbol,wasysym} % smileys
\usepackage{minibox} % multiline fbox - \minibox[frame]{}
\usepackage[pscoord]{eso-pic} % floating text box \placetextbox{x}{y}{text}
\usepackage{ifthen}
\usepackage{soul} % teste barré \st

% --- geometry ---
\usepackage{geometry}
\geometry{legalpaper, margin=1cm}
% ---

% --- xcolor ---
\usepackage{xcolor}
\definecolor{lightgray}{gray}{0.9}
% ---

% --- tcolorboxes ---
\usepackage[most]{tcolorbox}
\newtcolorbox{definition}[2][]{%
  attach boxed title to top left
               = {yshift=-8pt},
  colback      = white,
  colframe     = gray,
  fonttitle    = \bfseries,
  colbacktitle = gray,
  title        = #2,#1,
  enhanced,
}
% ---

% --- array ---
\usepackage{array}
\newcolumntype{L}[1]{>{\raggedright\let\newline\\\arraybackslash\hspace{0pt}}m{#1}}
\newcolumntype{C}[1]{>{\centering\let\newline\\\arraybackslash\hspace{0pt}}m{#1}}
\newcolumntype{R}[1]{>{\raggedleft\let\newline\\\arraybackslash\hspace{0pt}}m{#1}}
 % ---


\renewcommand{\baselinestretch}{1.15} % augmenter l'interligne

\dashundergapssetup{
	teacher-gap-format=underline,
	gap-widen
}

\newcommand{\placetextbox}[3]{% \placetextbox{<horizontal pos>}{<vertical pos>}{<stuff>}
  \setbox0=\hbox{#3}% Put <stuff> in a box
  \AddToShipoutPictureFG*{% Add <stuff> to current page foreground
    \put(\LenToUnit{#1\paperwidth},\LenToUnit{#2\paperheight}){\vtop{{\null}\makebox[0pt][c]{#3}}}%
  }%
}%


\author{Paul Clavier}
\title{Chapitre 1 - Tableaux et graphiques - Interrogation Écrite} 

\begin{document}

% --- Section & subsection renum ---
\renewcommand\thesection{\Roman{section}}
\renewcommand\thesubsection{\arabic{subsection}}
% ---

% --- Selection manuelle de la version ---
%\TeacherModeOn
% ---

% --- Selection automatique de la version ---
\ifdefined\isprof
	\TeacherModeOn
\fi

% ---

\begin{center}
	\minibox[frame,c]{Chapitre 2 - Les nombres décimaux \\ Interrogation Écrite}
\end{center}

\placetextbox{0.1}{0.99}{Nom:}
\placetextbox{0.1}{0.96}{Prénom:}

\begin{center}
\begin{tabular}{|l|c|}
\hline \rowcolor{lightgray}
Les nombres décimaux \hspace{8cm} & Maitrise \\ \hline
\thead[l]{1.3 : Utiliser et représenter les grands nombres entiers, des fractions simples, des nombres décimaux} &
\\ \hline
\thead[l]{2 : Présenter son travail de façon soignée} &
 \\ \hline
\end{tabular}
\end{center}

\textbf{Exercice 1}: Complète le tableau\\

\begin{tabular}{|c|c|c|c|c|}
\hline 
Nombre & Partie entière & Partie décimale & Chiffre des dixièmes & Nombre de centièmes \\ 
\hline 
45,12 & \gap*[b]{45} & \gap*[b]{0,12} & \gap*[b]{1} & \gap*[b]{4 512} \\ 
\hline 
1,789 & \gap*[b]{1} & \gap*[b]{0,789} & \gap*[b]{7} & \gap*[b]{178} \\ 
\hline 
0,475 & \gap*[b]{0} & \gap*[b]{0,475} & \gap*[b]{4} & \gap*[b]{75} \\ 
\hline 
14,5 & \gap*[b]{14} & \gap*[b]{0,5} & \gap*[b]{5} & \gap*[b]{50} \\ 
\hline 
\end{tabular}\\

\textbf{Exercice 2}: Écris chaque nombre sous forme de fraction décimale\\

\begin{minipage}{0.5\textwidth}
$14,8 = \frac{\text{\gap*[b]{148}}}{\text{\gap*[b]{10}}}$\\\\
$74,21 = \frac{\text{\gap*[b]{7 421}}}{\text{\gap*[b]{100}}}$\\\\
\end{minipage}
\begin{minipage}{0.5\textwidth}
$0,44 = \frac{\text{\gap*[b]{44}}}{\text{\gap*[b]{100}}}$\\\\
$665,975 = \frac{\text{\gap*[b]{665 975}}}{\text{\gap*[b]{1000}}}$\\\\
\end{minipage}

\textbf{Exercice 3}: Supprime les zéros inutiles
\begin{center}
\ifdefined\isprof
\st{0}1,1\st{0} - \st{000}475,707\st{0} - \st{0}0,975\st{00} - \st{0}801,108\st{0}
\else
01,10 - 000475,7070 - 00,97500 - 0801,1080
\fi
\end{center}

\textbf{Exercice 4}: Convertis les longueurs suivantes\\

\begin{minipage}{0.3\textwidth}
14cm = \gap*{140}mm\\
17m = \gap*{1 700}cm\\
21km = \gap*{21 000 000}mm
\end{minipage}
\begin{minipage}{0.3\textwidth}
5 100mm = \gap*{5,1}m\\
32cm = \gap*{0,32}m\\
192mm = \gap*{0,000 192}mm
\end{minipage}
\begin{minipage}{0.3\textwidth}
0,19dam = \gap*{0,019}hm\\
17 640dm = \gap*{1,764}km\\
0,000 191km = \gap*{1,91}dm
\end{minipage}\\

\textbf{Exercice Bonus}: Place la virgule pour que 8 soit le chiffre des millièmes dans\\

\begin{tabular}{|C{4cm}|C{4cm}|C{4cm}|C{4cm}|}
\hline 
1 4 7 9 8 & 7 4 5 3 1 9 8 2 3 1 & 1 0 8 1 0 3 & 1 4 5 3 2 1 7 4 5 3 8 \\ 
\hline 
\end{tabular}\\

\textbf{Exercice Bonus 2}: Place la virgule pour que 8 soit le chiffre des dizaines de milliers dans\\

\begin{tabular}{|C{4cm}|C{4cm}|C{4cm}|C{4cm}|}
\hline 
1 2 3 4 5 7 8 9 & 4 1 2 3 & 7 1 3 2 1 4 & 7 4 1 2 3 6 4 8 9 \\ 
\hline 
\end{tabular}

\end{document}